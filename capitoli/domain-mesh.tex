\chapter{General architecture domain mesh}
In questo capitolo si descriverà il percorso che ha portato alla creazione dell'architettura domain mesh, un modello che ci permette di soddisfare pienamente i requisiti di resilienza. La prima parte riguarda la scelta della topologia in base alle esigenze delle tecnologie usate (Percona, Liqo...), la seconda parte riguarderà come si possa sfruttare la topologia scelta per estendere i casi d'uso e nell'ultimo segmento si discuteranno i vantaggi/svantaggi

\section{Physical/net architecture}
spiegazione del perchè si debba usare una specie di full mesh e non approccio gerarchico

\section{Logic hierarchy with labels and affinity}
come ottenere la gerarchia logica usando labels and  affinità

\subsection{Goup indipendent/dependent}
spiegazione e figure di questo approccio gerarchico

\subsection{level}
spiegazione e figura di questo approccio gerarchico

\section{Domain mesh}
spiegazione e vantaggi + figura finale