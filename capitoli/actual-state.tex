\chapter{State of art}
In questo capitolo si espone l'implementazione corrente per supportare i paradigmi di edge e fog computing nella rete di produzione e monitoraggio dell'energia, basata sul modello Smart Grid(traslare i componenti hardware in una rete informatica)e sull'utilizzo della piattaforma Kubernetes, e si evidenzieranno i limiti che l'implementazione studiata in questa tesi andrà ad eliminare.

\section{Smart Grid components}
Correntemente, la rete di monitoraggio dell'energia è composta dall'Area control center, nodo centrale decisionale, e dalle stazioni di produzione e distribuzione, suddivise in primarie e secondarie. Per gestire la rete si utilizzano principalmente tre applicativi, Phasor Measurement Units (pmu) per le misurazioni, Phasor Data Concentrator del progetto openPDC (pdc) per l' aggregazione dei dati delle varie pmu e Grid State Estimation (gse) per monitorare la rete partendo dai dati forniti dalle precedenti applicazioni.

\subsection{Area control center}
Nodo informatico che rappresenta un Centro Operativo di distribuzione, quindi sede della logica di controllo e gestionale dell'intera rete. Questo nodo usualmente gestisce il pdc di alto livello, in cui si aggregano i flussi di dati provenienti dagli altri PDC di alto livello, e l'applicazione gse che utilizzerà i dati del precedente PDC per controllare la rete.

\subsection{Station}
Nodo informatico che rappresenta una stazione di produzione o distribuzione dell'energia. Le stazioni primarie utilizzano principalmente un PDC di alto livello per aggregare i flussi provenienti dalla sottorete di stazioni secondarie, ma possono gestire anche alcune PMU. Le stazioni secondarie principalmente si occupano di gestire le pmu oppure di di aggregare flussi minori attraverso pdc di basso livello. 

\subsection{Phasor Measurement Units}
Le PMU forniscono misurazioni di grandezze elettriche fondamentali, come tensione e corrente, nella forma di fasori, includendo informazioni sull'ampiezza e sulla fase delle grandezze misurate. Queste misurazioni, sincronizzate tramite GPS e campionate a una frequenza di 50 campioni al secondo, consentono il monitoraggio preciso dei rapidi cambiamenti nel sistema elettrico causati dalla dinamicità delle risorse energetiche distribuite.

Le PMU offrono una prospettiva dettagliata della dinamica del sistema, superando le limitazioni delle più tradizionali Remote Terminal Units (RTU), le quali presentano un periodo di aggiornamento di diversi secondi e non sono sincronizzate. L'impiego delle PMU è atteso migliorare l'osservabilità e l'affidabilità del sistema di distribuzione.

\subsection{Phasor Data Concentrator}
Un concentratore di dati fasoriali è progettato per ricevere dati di sincrofasori in streaming dalle unità di misura fasoriali (PMU) installate sulle linee di trasmissione dell'energia e allineare questi dati tramite tag temporale GPS (cioè, "concentra" i dati in base al tempo). L'output di un PDC è un set di dati sincronizzato nel tempo che viene inoltrato a una o più applicazioni software.

openPDC è una piattaforma flessibile per l'elaborazione di dati in serie temporale ad alta velocità, sia in tempo reale che storici. Non ha grossi requisiti in termini di potenza di calcolo, perciò può essere installato ovunque all'interno dell'infrastruttura dei sincrofasori, anche sui computer senza ventole presenti nelle sottostazioni.

\subsection{Grid State Estimation}
La stima dello stato della rete è una tecnica che consente la ricostruzione degli stati della rete, come ad esempio le tensioni nodali, basandosi sulle misurazioni disponibili e sul modello della rete elettrica. A differenza dei misuratori tradizionali, le misure delle PMU, includendo la fase rispetto a un riferimento assoluto, semplificano il problema della stima dello stato rendendolo un sistema lineare e riducendo notevolmente il carico computazionale. Gli obiettivi della stima dello stato comprendono il riconoscimento e la riduzione degli errori di misura, la individuazione di errori nella topologia, la stima di grandezze di rete non acquisite e la determinazione dei parametri di rete attraverso misure ridondate.

\section{Multi-master station architecture}
The current state of research has advanced to managing a single node/place of the Smart Grid through a multi-master architecture (CITA SEBASTIANO TESI e rticolo), leveraging the resilience gained from the ability to withstand the failure of a master node, albeit at the cost of increased resources required for replicated control-plane components.

Nel cluster la configurazione delle applicazioni (ad esempio la configurazione di un pdc nel caso il cluster rappresenti un stazione) viene salvata in un sistema di database distribuito high availability. Questo permette il veloce e automatico rideployamento in un altro nodo del cluster in caso di guasto, senza dover riconfigurare i parametri per quell'applicativo.
In questo modo i cluster supportano la resoluzione automatica locale dei guasti sia degli applicativi sia dei nodi in cui essi vengono deployati.

In base alla posizione, ogni cluster rappresenta nella Smart Grid un punto di edge (se gestisce una stazione secondaria) o fog (se gestisce stazione primaria) computing, ma l'architettura di controllo generale viene stabilita manualmente tra ogni coppia di cluster, i quali esistono come entità separate completamente indipendenti tra di loro.

\section{Actual challenge}
While this approach of a multi-cluster kubernetes is effective for managing a single station, it proves suboptimal when applied to an entire electrical control system. This is due to both the complexity involved in managing a large number of nodes (with stations alone numbering in the tens of thousands, whereas Kubernetes officially supports up to 5000 nodes (CITA DOCS?)) and the fundamental inability to function in isolation. In fact, if a segment of the network underlying a master node becomes isolated from the rest of the architecture, it becomes unmanageable as the master node loses the necessary consensus to initiate new workloads (new pods to manage the isolated entities) and can only partially manage existing workloads (because it can't reschedule workloads if it fails).
The only additional failure scenario that the translated architecture can address is when an entire secondary station becomes isolated from the network, allowing the possible PDC pod affected to be relocated to another station. However, this advantage does not outweigh the drawbacks in terms of complexity,lack of scalability, and resource demands inherent in the overall architecture. These challenges can be effectively addressed by adopting Liqo technology.