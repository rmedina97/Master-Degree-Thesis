\chapter{Domain peering evaluation}
In questo capitolo si mostrano le analisi fatte sull'implementazione Logical domain grouping, scelta poichè ha un grado di resilienza maggiore rispetto all'implementazione multi-level.
Si noti che alcuni parametri del k3s kubelet sono stati modificati per diminuire il tempo di risposta del cluster in caso di guasto, piu precisamente:
 (MA VIRTUAL KUBELET PRENDE QUESTI PARAMETRI?)
1)node-status-update-frequency=10s -> 5s 
Specifies how often kubelet posts node status to master

2)node-monitor-period=5s ->5s 
The period for syncing NodeStatus in cloud-node-lifecycle-controller

3)node-monitor-grace-period=40s -> 20s
Amount of time which we allow running Node to be unresponsive before marking it unhealthy. Must be N times more than kubelet's nodeStatusUpdateFrequency, where N means number of retries allowed for kubelet to post node status.

4)pod-eviction-timeout=300s -> 5s
Questo parametro specifica quanto tempo deve passare prima che i pod vengano sfrattati da un nodo segnato come "NotReady


\section{latency}
In questa sezione si mostra l'incremento della latenza dovuto all'overhead generato dalla tecnologia Liqo, utilizzando 5 virtual machines (un radice e 4 foglie) presenti in cluster diversi. Questo incremento viene misurato sottraendo la latenza misurata tra le due macchine virtuale, appartenenti a due cluster differenti, alla latenza misurata tra due pod presenti sulle stesse. Questo poichè la latenza tra due macchine virtuali è semplicemente la latenza della rete, mentre la latenza tra due pod è la latenza della rete + overhead di Liqo, in quanto il traffico passa attraverso il tunnel Liqo. Overhead di Kubernetes non entra in gioco in quanto il traffico viene gestito immediatamente da Liqo e non da Kubernetes

Considerando che i dati nella nostra architettura sono spediti utilizzando protocolli TCP, la quale utilizza gli ack, si considera come latenza il round trip time di un pacchetto. I seguenti test sono stati effetuati utilizzando il comand ping, per un totale di circa 1000 iterazioni per test.

La latenza tra due macchine è stata misurata come latenza media tra le 4 latenze (METTERE LATENZE E METTERE FORMULA?)
0.665 ± 0.662 ms 
0.689 ± 0.447 ms 
0.760 ± 0.771 ms 
0.715 ± 0.468 ms
ottenendo 0.707 ± 0.595 ms (mettere immagine invece di scrivere dati?)

la latenza tra due pod è stata misurata come latenza media tra le 4 latenze
1.174 ± 0.670 ms
1.269 ± 0.724 ms
1.368 ± 0.903 ms
1.400 ± 0.829 ms
ottenendo 1.302 ± 0.691 ms

La latenza architetturale aggiuntiva si ottiene sottraendo la latenza media tra le macchine virtuali dalla latenza media tra i pod.

1.302 ± 0.691 ms - 0.707 ± 0.595 ms = 0.595 ± 0.932 ms

\section{k3s reaction time}
spiegazione e grafici, comparazione con un normale kubernetes, valori minimi
Questi test rappresentano il tempo medio->rifallo con parametri uguali ma valutando caduta nodo liqo e caduta nodo senza liqo

\section{Stream reaction time}
I seguenti test mostrano il tempo di downtime dello stream di dati creato dalla catena di PDC nei seguenti casi:
1)guasto interno al pod PDC
2)guasto/disconnessione del cluster in cui era presente un pod PDC 
Downtime is calculated from the timestamp of the last data frame of the old stream to the timestamp of the first data frame of the new stream.

\subsection{Pod failure}
Qui si mostra il caso in cui si guasta per problemi interni il pod, portando alla ricreazione dello stesso 
\subsection{Cluster failure}
Qui si mostra il caso in cui si perde la connessione con il cluster contenete il pod, perciò c'è il tempo di reaction time di k3s + la creazione di un nuovo pod


