\chapter{Possible implementations}
In questo capitolo saranno discusse le possibili implementazioni della topologia a stella a maglia parziale nel contesto di una rete informatica adibita al monitoraggio dell'energia, composta principalmente dall'Area Control Center, stazioni primarie e stazioni secondarie, ognuna gestita dal proprio cluster kubernetes.

\section{Logical domain grouping}
L'Area Control Center occupa la posizione centrale della topologia,  stabilendo un peering unidirezionale con offloading con ogni altra entità della topologia, che sia una stazione primaria che una stazione secondaria. Questo permette all'Area Control Center di poter gestire tutti i deployment delle applicazioni, senza doverli delegare ad altri nodi.
Il resto dei cluster viene suddiviso in gruppi senza collegamenti tra di loro, formati usualmente da una stazione primaria e dalle stazione secondarie a cui fa capo. Questi gruppi rappresentano un dominio logico delle applicazioni con un proprio sistema di database distribuito HA, stabilendo perciò una mesh parziale tra i nodi del gruppo e condividendo lo stesso namespace offlodato dal cluster centrale. 
Questa architettura permette il massimo grado di resilienza, poichè l'unico punto critico è l'Area Control Center e gli effetti della caduta di questo nodo sono trascurabili se rapportati ai vincoli ambientali:  
1)In caso di guasto fisico del nodo centrale si perderebbero i deployment, rendendo impossibile il processo di riconciliazione con l'intera rete, ma questo è trascurabile poichè con la caduta della logica del nodo centrale non si potrebbe osservare la rete di default
2)In caso di completa disconnessione dalla rete, i carichi di lavoro attivi continuerebbero a funzionare ma il processo di riconciliazione delle applicazioni stateful non potrebbe avvenire in quanto dal punto di vista del deployment non è sopravvisuto il numero necessario per mantenere il sistema. Ma anche in questo caso è trascurabile, in quanto ricadremmo nello stesso caso di prima
Al contrario guasti o disconessioni sui cluster foglia sono del tutto supportati, in quanto dal punto di vista del nodo centrale il loro carico viene semplicemente spostato su altri cluster affini, mentre nei cluster foglia, se sono solo disconnessi, i carichi di lavoro continuano a funzionare con la possibilità di istanziare nuovi applicativi per permettere il funzionamento ad isola fino alla riconciliazione.
I limiti di questa architettura riguardano la scalabilità, poichè per il funzionamento trasparente dei sistemi di database distribuiti HA ogni cluster ha bisogno della propria CIDR distinta ma ogni peering crea nel cluster centrale un nodo virtuale di rappresentanza, ma nel caso di migliaia di cluster si potrebbe superare il limite consigliato da kubernetes di 5000 nodi (RIFLETTI MEGLIO SU QUESTO ASPETTO)

\section{Multi-level logical domain grouping}
Questa implementazione sfrutta due volte la topologia stella a maglia parziale/completa, seguendo la suddivisione delle stazioni in primarie e secondarie, ma potrebbe essere adattata a n suddivisioni (CITA  verifica cio).
La prima topologia è una stella a maglia parziale utilizzata per connettere l'Area Control Center (cluster centrale) con tutte le stazioni primarie (cluster foglie). Il cluster centrale si occupa del deployment delle applicazioni di alto livello con il loro relativo sistema di database distribuito, offlodando il corrispettivo namespace attraverso i peering al corrispettivo gruppo di stazioni primarie. I gruppi di stazioni primarie son composte da una stazione primaria principale, in cui preferibilmente finirà il carico di lavoro (attraverso l'uso di labels), mentre le altre presenti nel gruppo hanno principalmente funzione di backup in caso di guasto della primaria.
La primaria di ogni gruppo è anche il cluster centrale della seconda topologia a stella, avendo oltre al collegamento con le stazioni primarie di backup tutti i collegamenti con le stazioni secondarie di sua competenza. Sarà una topologia a stella completa se le stazioni secondarie condivideranno tutte lo stesso dominio di DB, a maglia parziale in caso contrario. 
La stazione primaria si occupa del deployment delle applicazioni di basso livello con il loro relativo database distribuito HA, offlodando di conseguenza il namespace alle sue stazioni secondarie. Lo stream di dati per il monitoraggio, dovendo passare attraverso due namespace diversi (da quello basso a quello alto) si appoggia a servizi di esposizione come load balancer o ingress, permettendo di raggiungere l'applicativo di alto livello sia che si trovi nella stazione primaria sia che si trovi a causa di un guasto in una delle stazioni primarie di backup.
(PARTE DA CHIEDERE PER IL LOAD BALANCER ANCHE SE CREDO NON SERVA, DEVO RIVEDERE IL TUTTO)
Questa architettura permette di abbassare i limiti alla scalabilità della precedente implementazione poichè, oltre ad abbassare il numero di peering gestiti dall'Area Control Center, i CIDR dei cluster dovranno essere distinti solo all'interno delle topologie secondarie che fanno a capo di una stazione primaria, permettendone il riutilizzo nelle diverse tipologie. Questo benefit viene bilanciato dall'abbassamento generale del grado di resilienza, poichè in caso di guasto o disconnessione della stazione primaria si perde il deployment degli applicativi di basso livello, in quanto non è una risorsa riflessa dalla tecnologia Liqo, portando al reset del sistema in caso di riconnessione.

\section{Final consideration}
Questa tesi è focalizzata sull'ottenere il massimo grado di resilienza, perciò nei successivi capitoli ci si occuperà di testare la prima implementazione. Si ricorda solo che queste due implementazioni non sono mutualmente esclusive, possono essere implementate contemporaneamente nella stessa rete fisica, nel caso diverse parti della rete abbiano bisogno di diversi gradi di resilienza.